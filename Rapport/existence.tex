 \documentclass[onecolumn, 12pt, a4paper]{article}
\usepackage[utf8]{inputenc}
\usepackage{amsmath,amssymb}
\usepackage[utf8]{inputenc}
\usepackage{fontenc}
\usepackage{graphicx}
\usepackage{fancyhdr}
\usepackage{xcolor}
\usepackage{lipsum}
\usepackage{fancybox}
\usepackage{float}
\usepackage{biblatex}

\usepackage{mdframed}
\usepackage{setspace}
\usepackage{geometry} 
\usepackage[french]{babel}

\geometry{hmargin=1.5 cm,vmargin=2cm}
\linespread{1.25}

\begin{document}
	
	
	\begin{center}
		
		
		\vspace{50pt}
		%\includegraphics[scale=0.5]{logo_em.jpg}\\
		\vspace{20pt}
		
		\rule{1\textwidth}{2pt}\\
		\vspace{20pt}
		\huge Existence des solutions\\
		\vspace{20pt}
		\rule{1\textwidth}{2pt}\\ 
		\vspace{10pt}
		
	\end{center}
	
	\noindent
	\begin{minipage}{0.5\textwidth}
		\begin{flushleft}
			\Large Writted by :\\[0.2cm]
			\textsc{El Maddah} Khaloula\\
			\textsc{Doyen} Guillaume\\
			\textsc{Rodiere} Gabriel\\
			\textsc{Boucher} Antoine\\
			\textsc{Aumonier} Clément\\
		\end{flushleft}
	\end{minipage}%
	\begin{minipage}{0.5\textwidth}
		\begin{flushright}
			\Large Supervised by :\\[0.2cm]
			\textsc{Poëtte} Gaël\\
			
		\end{flushright}
	\end{minipage}
	
	\vspace{50 pt}
	
	
	
	
	
	\newpage
	\footnotesize
	
	\tableofcontents
	\setcounter{tocdepth}{1}
	\setcounter{tocdepth}{0}
	\def\restriction#1#2{\mathchoice
		{\setbox1\hbox{${\displaystyle #1}_{\scriptstyle #2}$}
			\restrictionaux{#1}{#2}}
		{\setbox1\hbox{${\textstyle #1}_{\scriptstyle #2}$}
			\restrictionaux{#1}{#2}}
		{\setbox1\hbox{${\scriptstyle #1}_{\scriptscriptstyle #2}$}
			\restrictionaux{#1}{#2}}
		{\setbox1\hbox{${\scriptscriptstyle #1}_{\scriptscriptstyle #2}$}
			\restrictionaux{#1}{#2}}}
	\def\restrictionaux#1#2{{#1\,\smash{\vrule height .8\ht1 depth .85\dp1}}_{\,#2}} 
	
	
	\newpage
	
	\paragraph{Note} Here is a brief summary of the various existences and uniqueness of solutions based on different cases of equations to be solved. The primary source used is the transport and diffusion document.
	
	\section{Transport Equation}
	
	\paragraph{}
	In this section, we will examine the solutions of the transport equation, a particular case of the Boltzmann equation, defined below:
	
	
	
	Let's $v \in \mathbb{R}^N$
	
	\begin{equation} \label{transport}
		\frac{\partial f}{\partial t}(t,x)+v \cdot \nabla_x f(t,x) + a(t,x)f(t,x) = S(t,x), \quad x \in \mathbb{R}^N, \quad t>0.
	\end{equation}
	
	
	$f$ is the unknown function, $a$ is the given damping or amplification rate, and $S$ is the given source term.
	
	\subsection{Cauchy Problem}
	\subsubsection{Special Case without Absorption or Source Term}
	
	Let's consider the following Cauchy problem:
	
	\[
	\begin{cases}
		\frac{\partial f}{\partial t}(t,x)+v \cdot \nabla_x f(t,x)=0, \quad x \in \mathbb{R}^N, \quad t>0.\\
		f(0,x)) = f^{in}(x)
	\end{cases}
	\]
	
	If $f^{in} \in \mathcal{C}^1(\mathbb{R}^n)$, then using the method of characteristics, we can easily demonstrate that the Cauchy problem has a unique solution $f \in \mathcal{C}^1(\mathbb{R}_+ \times \mathbb{R}^n)$. This solution is given as follows:
	
	
	
	
	
	$\boxed{f(t,x)=f^{in}(x-tv) \quad \forall (t,x) \in \mathbb{R}_+ \times \mathbb{R}^n }$
	
	
	
	
	\subsubsection{With Absorption and Source Term}
	
	This time, we consider damping and the source term, assuming they belong to $\mathcal{C}^1(\mathbb{R}_+ \times \mathbb{R}^n)$. The considered Cauchy problem is as follows:
	
	
	\[
	\begin{cases}
		\frac{\partial f}{\partial t}(t,x)+v \cdot \nabla_x f(t,x) + a(t,x)f(t,x) = S(t,x), \quad x \in \mathbb{R}^N, \quad t>0.\\
		f(0,x)) = f^{in}(x)
	\end{cases}
	\]
	
	In this case, if $f^{in} \in \mathcal{C}^1(\mathbb{R}^n)$, then using the method of characteristics, we can still demonstrate that the Cauchy problem has a unique solution $f \in \mathcal{C}^1(\mathbb{R}_+ \times \mathbb{R}^n)$. This solution is given as follows:\\
	$\boxed{f(t,x)=f^{in}(x-tv) e^{-\int_0 ^t a(\tau,x+(t-\tau)v)d\tau} + \int_0 ^t  e^{-\int_0 ^t a(\tau,x+(t-\tau)v)d\tau} S(s,x+(s-t)v)ds \quad \forall (t,x) \in \mathbb{R}_+ \times \mathbb{R}^n }$
	
	\subsection{Boundary Value Problems}
	
	In the previous section, we presented results applicable when considering complete freedom for the spatial variable in $\mathbb{R}^n$. However, for a significant portion of physical problems, this may not be the case. That is why we are more interested in a problem of the boundary value type. In this section, we will consider the set $\Omega$, which is an open set with a $\mathcal{C}^1$ boundary in $\mathbb{R}^n$. Additionally, we take $v \in \mathbb{R}^n\backslash \{0\}$.
	
	\subsubsection{Special Case without Absorption or Source Term}
	
	\paragraph{}
	If, moreover, $\Omega$ is convex, and let $f_b^- \in \mathcal{C}^1(\mathbb{R}_+ \times \partial \Omega^-)$ and $f^{in} \in \mathcal{C}^1(\overline{\Omega})$, where $\partial \Omega^- = \{x\in\partial \Omega~ | ~ v\cdot n_x<0\}$, then the problem:
	
	
	\[
	\begin{cases}
		\frac{\partial f}{\partial t}(t,x)+v \cdot \nabla_x f(t,x)=0, \quad x \in \mathbb{R}^N, \quad t>0.\\
		\restriction{f}{\partial \Omega}= f_b^-\\
		\restriction{f}{t=0}= f^{in}(x)
	\end{cases}
	\]
	
	has a unique solution $f \in \mathcal{C}^1(\mathbb{R}_+ \times \Omega)$ and $f \in \mathcal{C}^0(\mathbb{R}_+ \times \overline{\Omega})$ if and only if $\forall y \in \partial \Omega^-,$
	
	\[ f_b^-(0,y) = f^{in}(y), \quad \frac{\partial f_b^-}{\partial t}(t,y)+v \cdot \nabla f^{in}(y)=0. \]
	
	\paragraph{}
	
	The expression of the solution is then given by: 
	\[
	\boxed{
		f(t,x)= 
		\begin{cases}
			f^{in}(x-tv) \quad si~ t \leq \tau_x ,\\
			f_b^-(t-\tau_x,x-\tau_xv) \quad si~ t > \tau_x ,
		\end{cases}
	}
	\]
	
	With $\tau_x= \inf \{t \geq 0~ | ~ x-tv \notin \overline{\Omega}\}$
	
	\subsubsection{With Absorption and Source Term}
	
	\paragraph{}
	We still consider $\Omega$ to be convex, with $f_b^- \in \mathcal{C}^1(\mathbb{R}_+ \times \partial \Omega^-)$, $f^{in} \in \mathcal{C}^1(\overline{\Omega})$, $a$, and $S$ $\in \mathcal{C}^1(\mathbb{R}_+ \times \overline{\Omega})$. If we assume that $\forall y \in \partial \Omega^-$,
	
	\[
	\begin{aligned}
		f_b^-(0,y) &= f^{in}(y), \\
		\frac{\partial f_b^-}{\partial t}(0,y) + v \cdot \nabla_x f^{in}(y) + a(0,y)f^{in}(y)&=S(0,y)
	\end{aligned}
	\]
	
	
	
	In this case, the problem becomes:
	
	\[
	\begin{cases}
		\frac{\partial f}{\partial t}(t,x)+v \cdot \nabla_x f(t,x)+ a(t,x)f(t,x)=S(t,x), \quad x \in \mathbb{R}^N, \quad t>0.\\
		\restriction{f}{\partial \Omega}= f_b^-\\
		\restriction{f}{t=0}= f^{in}(x)
	\end{cases}
	\]
	
	has a unique solution $f \in \mathcal{C}^1(\mathbb{R}_+ \times \Omega)$ and $f \in \mathcal{C}^0(\mathbb{R}_+ \times \overline{\Omega})$. The expression for $f$ is then:
	
	
	\[
	\boxed{
		\begin{aligned}
			f(t,x) &= \mathbf{1}_{t \leq \tau_x} f^{in} (x-tv) \exp\left(-\int_0^t a(s,x+(s-t)v)ds\right) \\
			&\quad + \mathbf{1}_{t > \tau_x} f_b^- (t-\tau_x,x-\tau_x v) \exp\left(-\int_{t-\tau_x}^t a(s,x+(s-t)v)ds\right) \\
			&\quad + \int_{t-\tau_x}^t \exp\left(-\int_{s}^{t}a(\tau,x+(\tau-t)v)d\tau\right) S(s,x+(s-t)v)ds
	\end{aligned}}
	\]
	
	
	\subsection{Generalized Solutions}
	
	In this section, $\Omega$ is not necessarily convex. This implies that the continuity of potential solutions is not guaranteed. However, in practice, these cases must be addressed. Below, we define the concept of generalized solutions, which allows us to handle this issue without resorting to the theory of distributions.
	
	
	\paragraph{Definition}
	
	Let $\Omega$ be an open set with a $\mathcal{C}^1$ boundary in $\mathbb{R}^n$, and let $a$ and $S$ be continuous on $\mathbb{R}_+ \times \Omega$. We say that $f$ is a generalized solution of the problem
	
	$$\frac{\partial f}{\partial t}(t,x)+v \cdot \nabla_x f(t,x)+ a(t,x)f(t,x)=S(t,x), \quad x \in \mathbb{R}^N, \quad t>0.\\$$
	
	if and only if the function
	
	$s \mapsto f(t+s,x+sv)$ is of class $\mathcal{C}^1$ almost everywhere in $(t,x) \in \mathbb{R}_+ \times \Omega$ and satisfies
	
	$\frac{d}{ds}f(t+s,x+sv) +a(t+s,x+sv)f(t+s,x+sv) = S(t+s,x+sv) \quad (t+s,x+sv) \in \mathbb{R}_+^* \times \Omega$.
	
	
	\paragraph{}
	
	
	
	Let $\Omega$ be an open set with a $\mathcal{C}^1$ boundary in $\mathbb{R}^n$, and let $a$ and $S$ be continuous on $\mathbb{R}_+ \times \Omega$. Let $f_b^- \in \mathbf{L}_{loc}^{\infty}(\mathbb{R}_+ \times \partial \Omega^-)$ and $f^{in} \in \mathbf{L}_{loc}^{\infty}(\overline{\Omega})$.
	
	In this case, the problem:
	
	\[
	\begin{cases}
		\frac{\partial f}{\partial t}(t,x)+v \cdot \nabla_x f(t,x)+ a(t,x)f(t,x)=S(t,x), \quad x \in \mathbb{R}^N, \quad t>0.\\
		\restriction{f}{\partial \Omega}= f_b^-\\
		\restriction{f}{t=0}= f^{in}(x)
	\end{cases}
	\]
	
	has a unique generalized solution. The formula for this solution is the same as in the previous case, i.e., almost everywhere in $(t,x) \in \mathbb{R}_+ \times \Omega$, we have:
	
	\[
	\boxed{
		\begin{aligned}
			f(t,x) &= \mathbf{1}_{t \leq \tau_x} f^{in} (x-tv) \exp\left(-\int_0^t a(s,x+(s-t)v)ds\right) \\
			&\quad + \mathbf{1}_{t > \tau_x} f_b^- (t-\tau_x,x-\tau_x v) \exp\left(-\int_{t-\tau_x}^t a(s,x+(s-t)v)ds\right) \\
			&\quad + \int_{t-\tau_x}^t \exp\left(-\int_{s}^{t}a(\tau,x+(\tau-t)v)d\tau\right) S(s,x+(s-t)v)ds
	\end{aligned}}
	\]
	
	
	
	
	
	
	
	\section{Linear Boltzmann Equation}
	
	In this section, we will focus on the actual Boltzmann equation. This integro-differential equation is described in \ref{Boltzmann}.
	
	\begin{equation} \label{Boltzmann}
		\frac{\partial f}{\partial t}(t,x,v)+v \cdot \nabla_x f(t,x,v) + a(t,x)f(t,x,v) = \int_{\mathbb{R}^n}k(t,x,v,w)f(t,x,w)d\mu(w)+Q(t,x,v).
	\end{equation}
	
	To simplify notations, this equation will be written as: 
	\[ \frac{\partial f}{\partial t}+v \cdot \nabla_x f +af =   \mathcal{K} f +Q \]
	
	Here, $v$ and $k$ are given continuous functions. The range of possibilities for the velocity $v$ is very diverse due to the variety of physical problems that can be considered. Therefore, the notation $d\mu(w)$ encompasses all these possibilities by incorporating a weight function. The major difference from the transport equation is that, this time, we consider exchange phenomena, and thus, $v$ cannot be seen as a simple parameter.
	
	In this section, we cannot highlight the exact expression of the solutions because the method of characteristics yields another integral equation involving the solution.
	
	
	
	
	\paragraph{Definition}
	
	Let $Q$ be continuous on $]0,T[\times \Omega \times \mathbb{R}^n$. We say that $f$ is a generalized solution of the problem, continuous on $]0,T[\times \Omega \times \mathbb{R}^n$,
	
	
	$$\frac{\partial f}{\partial t}+v \cdot \nabla_x f +af = \mathcal{K} f +Q\\$$
	
	\paragraph{}
	if and only if, for all $(t,x,v) \in ]0,T[\times \Omega \times \mathbb{R}^n$, the function 
	
	\[ s \mapsto f(t+s,x+sv,v) \]
	
	is of class $\mathcal{C}^1$ for all $x+sv \in \Omega$ and satisfies, for all $s \in \mathbb{R}$ such that $x +sv \in \Omega$,
	
	$\frac{d}{ds}f(t+s,x+sv,v) +a(t+s,x+sv,v)f(t+s,x+sv,v) = (\mathcal{K}f+ Q)(t+s,x+sv,v).$
	
	\subsection{Cauchy Problem}
	\paragraph{}
	
	
	
	Let $f^{in} \in \mathbf{C}_b(\mathbb{R}^n \times \mathbb{R}^n)$ and $Q \in \mathbf{C}_b(]0,T[ \times \mathbb{R}^n \times \mathbb{R}^n)$. Also, let $0 \leq a \in \mathbf{C}_b(]0,T[ \times \mathbb{R}_x^n \times \mathbb{R}_v^n)$ and $0 \leq k \in \mathbf{C}_b(]0,T[ \times \mathbb{R}_x^n \times \mathbb{R}_v^n \times \mathbb{R}_w^n)$. The notation $\mathbf{C}_b(X)$ denotes the set of bounded continuous functions on $X$ with values in $\mathbb{R}$.
	
	
	We assume that:  \[ \underset{(t,x,v) \in ]0,T[ \times \mathbb{R}^n \times \mathbb{R}^n}{\sup} \int_{\mathbb{R}^n} k(t,x,v,w) \, d\mu(w) < +\infty \]
	
	\paragraph{}
	
	Then the problem is:
	
	\[
	\begin{cases}
		\frac{\partial f}{\partial t}+v \cdot \nabla_x f +af =   \mathcal{K} f +Q \quad (t,x,v) \in ]0,T[ \times \mathbb{R}^n \times \mathbb{R}^n\\
		\restriction{f}{t=0}= f^{in}
	\end{cases}
	\]
	has a unique generalized solution $f \in \mathbf{C}_b(]0,T[ \times \mathbb{R}^n \times \mathbb{R}^n)$.
	
	\subsection{Boundary Value Problem}
	
	We define the set $\Gamma_-$ as $\Gamma_-= \{(x,v) \in \partial \Omega \times \mathbb{R}^n ~ | ~ v\cdot n_x<0\}$
	
	Let $f^{in} \in \mathbf{C}_b(\overline{\Omega} \times \mathbb{R}^n)$, $f_b^- \in \mathbf{C}_b([0,T] \times \Gamma_-)$ such that $f_b^-(0,y,v)=f^{in}(y,v) \quad \forall (y,v) \in \Gamma_-$
	
	Let $Q \in \mathbf{C}_b([0,T] \times \overline{\Omega} \times \mathbb{R}^n)$.
	
	Let $0 \leq a \in \mathbf{C}_b(\mathbb{R}_+ \times \Omega_x \times \mathbb{R}_v^n)$ and $0 \leq k \in \mathbf{C}_b(\mathbb{R}_+ \times \Omega_x^n \times \mathbb{R}_v^n \times \mathbb{R}_w^n)$.
	
	Then the problem
	
	\[
	\begin{cases}
		\frac{\partial f}{\partial t}+v \cdot \nabla_x f +af =   \mathcal{K} f +Q \quad (t,x,v) \in ]0,T[ \times \Omega \times \mathbb{R}^n\\
		\restriction{f}{t=0}= f^{in}\\
		\restriction{f}{\Gamma^-}=f_b^-
	\end{cases}
	\]
	
	has a unique generalized solution $f \in \mathbf{C}_b([0,T] \times \Omega \times \mathbb{R}^n)$.
	
	
	
	
\end{document}

