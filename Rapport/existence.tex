 \documentclass[onecolumn, 12pt, a4paper]{article}
\usepackage[utf8]{inputenc}
\usepackage{amsmath,amssymb}
\usepackage[utf8]{inputenc}
\usepackage{fontenc}
\usepackage{graphicx}
\usepackage{fancyhdr}
\usepackage{xcolor}
\usepackage{lipsum}
\usepackage{fancybox}
\usepackage{float}
\usepackage{biblatex}

\usepackage{mdframed}
\usepackage{setspace}
\usepackage{geometry} 
\usepackage[french]{babel}

\geometry{hmargin=1.5 cm,vmargin=2cm}
\linespread{1.25}

\begin{document}


    \begin{center}


    \vspace{50pt}
    %\includegraphics[scale=0.5]{logo_em.jpg}\\
    \vspace{20pt}
    
    \rule{1\textwidth}{2pt}\\
    \vspace{20pt}
    \huge Existence des solutions\\
    \vspace{20pt}
    \rule{1\textwidth}{2pt}\\ 
    \vspace{10pt}

    \end{center}

    \noindent
    \begin{minipage}{0.5\textwidth}
        \begin{flushleft}
            \Large Writted by :\\[0.2cm]
            \textsc{El Maddah} Khaloula\\
            \textsc{Doyen} Guillaume\\
            \textsc{Rodiere} Gabriel\\
            \textsc{Boucher} Antoine\\
            \textsc{Aumonier} Clément\\
        \end{flushleft}
    \end{minipage}%
    \begin{minipage}{0.5\textwidth}
        \begin{flushright}
                \Large Supervised by :\\[0.2cm]
                \textsc{Poëtte} Gaël\\

        \end{flushright}
    \end{minipage}
 
	\vspace{50 pt}
 


 

\newpage
\footnotesize

\tableofcontents
\setcounter{tocdepth}{1}
\setcounter{tocdepth}{0}
\def\restriction#1#2{\mathchoice
	{\setbox1\hbox{${\displaystyle #1}_{\scriptstyle #2}$}
		\restrictionaux{#1}{#2}}
	{\setbox1\hbox{${\textstyle #1}_{\scriptstyle #2}$}
		\restrictionaux{#1}{#2}}
	{\setbox1\hbox{${\scriptstyle #1}_{\scriptscriptstyle #2}$}
		\restrictionaux{#1}{#2}}
	{\setbox1\hbox{${\scriptscriptstyle #1}_{\scriptscriptstyle #2}$}
		\restrictionaux{#1}{#2}}}
\def\restrictionaux#1#2{{#1\,\smash{\vrule height .8\ht1 depth .85\dp1}}_{\,#2}} 


\newpage

\paragraph{Remarque} Voici un petit résumé des différentes existences et unicités des solutions selon les différents cas d'équation à résoudre. La principale source utilisé est le document transport et diffusion.

\section{\textcolor{blue}{Equation de transport}}

\paragraph{}
Dans cette section nous allons étudier les solutions de l'équation de transport, un cas particulier de l'équation de Boltzmann, définie ci dessous : 



Soit $v \in \mathbb{R}^N$

\begin{equation} \label{transport}
\frac{\partial f}{\partial t}(t,x)+v \cdot \nabla_x f(t,x) + a(t,x)f(t,x) = S(t,x), \quad x \in \mathbb{R}^N, \quad t>0.
\end{equation}


$f$ est la fonction inconnue, $a$ est le taux d'amortissement ou d'amplification donné et $S$ est le terme source donné.

\subsection{Problème de Cauchy}
\subsubsection{Cas particulier sans absorption ni terme source}

Considérons le problème de Cauchy suivant:

\[
\begin{cases}
\frac{\partial f}{\partial t}(t,x)+v \cdot \nabla_x f(t,x)=0, \quad x \in \mathbb{R}^N, \quad t>0.\\
f(0,x)) = f^{in}(x)
\end{cases}
\]

Si $f^{in} \in \mathcal{C}^1(\mathbb{R}^n)$ alors par la méthode des caractéristiques on peut aisément montrer que le problème de Cauchy admet une unique solution $f \in \mathcal{C}^1(\mathbb{R}_+ \times \mathbb{R}^n)$. Cette solution est donnée comme étant: \\





$\boxed{f(t,x)=f^{in}(x-tv) \quad \forall (t,x) \in \mathbb{R}_+ \times \mathbb{R}^n }$




\subsubsection{Avec absorption et terme source}

Cette fois-ci nous considérons l'amortissement et le terme source et nous les supposons appartenir à $\mathcal{C}^1(\mathbb{R}_+ \times \mathbb{R}^n)$ Le problème de Cauchy considéré est le suivant:

	
\[
\begin{cases}
\frac{\partial f}{\partial t}(t,x)+v \cdot \nabla_x f(t,x) + a(t,x)f(t,x) = S(t,x), \quad x \in \mathbb{R}^N, \quad t>0.\\
f(0,x)) = f^{in}(x)
\end{cases}
\]

Dans ce cas si $f^{in} \in \mathcal{C}^1(\mathbb{R}^n)$ alors par la méthode des caractéristiques on peut toujours montrer que le problème de Cauchy admet une unique solution $f \in \mathcal{C}^1(\mathbb{R}_+ \times \mathbb{R}^n)$. Cette solution est donnée comme étant:\\
$\boxed{f(t,x)=f^{in}(x-tv) e^{-\int_0 ^t a(\tau,x+(t-\tau)v)d\tau} + \int_0 ^t  e^{-\int_0 ^t a(\tau,x+(t-\tau)v)d\tau} S(s,x+(s-t)v)ds \quad \forall (t,x) \in \mathbb{R}_+ \times \mathbb{R}^n }$

\subsection{Problèmes aux limites}



Dans la section précédente nous avons présenter des résultats applicable lorsque l'on considère une liberté totale pour la variable spatial dans $\mathbb{R}^n$. Or pour une majeur partie des problèmes physique ça ne sera pas le cas. C'est pour cela que nous nous intéressons d'avantage a un problème de type problèmes aux limites. Dans cette partie on va donc considérer l'ensemble $\Omega$ qui est un ouvert à bord de classe $\mathcal{C}^1$ de $\mathbb{R}^n$. De plus on prend $v \in \mathbb{R}^n\backslash \{0\}$

\subsubsection{Cas particulier sans absorption ni terme source}

\paragraph{}
Si de plus $\Omega$ est convexe et soient $f_b^- \in \mathcal{C}^1(\mathbb{R}_+ \times \partial \Omega^-)$ et $f^{in} \in \mathcal{C}^1(\overline{\Omega})$, où $\partial \Omega^- = \{x\in\partial \Omega~ | ~ v\cdot n_x<0\}$alors le problème :


\[
\begin{cases}
\frac{\partial f}{\partial t}(t,x)+v \cdot \nabla_x f(t,x)=0, \quad x \in \mathbb{R}^N, \quad t>0.\\
\restriction{f}{\partial \Omega}= f_b^-\\
\restriction{f}{t=0}= f^{in}(x)
\end{cases}
\]

admet une unique solution $f \in \mathcal{C}^1(\mathbb{R}_+ \times \Omega)$ et $f \in \mathcal{C}^0(\mathbb{R}_+ \times \overline{\Omega})$ si et seulement si $\forall y \in \partial \Omega^-,$
\\$  f_b^- (0,y) = f^{in}(y), \quad \frac{\partial f_b^-}{\partial t}(t,y)+v \cdot \nabla f^{in}(y)=0$.

\paragraph{}

L'expression de la solution est alors donnée par 
\[
\boxed{
f(t,x)= 
\begin{cases}
f^{in}(x-tv) \quad si~ t \leq \tau_x ,\\
f_b^-(t-\tau_x,x-\tau_xv) \quad si~ t > \tau_x ,
\end{cases}
}
\]

Avec $\tau_x= \inf \{t \geq 0~ | ~ x-tv \notin \overline{\Omega}\}$

\subsubsection{Avec absorption et terme source}

\paragraph{}
On considère toujours $\Omega$ convexe, $f_b^- \in \mathcal{C}^1(\mathbb{R}_+ \times \partial \Omega^-)$, $f^{in} \in \mathcal{C}^1(\overline{\Omega})$, $a$ et $S$ $\in \mathcal{C}^1(\mathbb{R}_+ \times \overline{\Omega})$. Si l'on suppose que $\forall y \in \partial \Omega^-$,

\[
\begin{aligned}
f_b^-(0,y) &= f^{in}(y), \\
\frac{\partial f_b^-}{\partial t}(0,y) + v \cdot \nabla_x f^{in}(y) + a(0,y)f^{in}(y)&=S(0,y)
\end{aligned}
\]



Dans ce cas le problème :

\[
\begin{cases}
\frac{\partial f}{\partial t}(t,x)+v \cdot \nabla_x f(t,x)+ a(t,x)f(t,x)=S(t,x), \quad x \in \mathbb{R}^N, \quad t>0.\\
\restriction{f}{\partial \Omega}= f_b^-\\
\restriction{f}{t=0}= f^{in}(x)
\end{cases}
\]

admet une unique solution $f \in \mathcal{C}^1(\mathbb{R}_+ \times \Omega)$ et $f \in \mathcal{C}^0(\mathbb{R}_+ \times \overline{\Omega})$. L'expresssion de $f$ est alors:



\[
\boxed{
\begin{aligned}
f(t,x) &= \mathbf{1}_{t \leq \tau_x} f^{in} (x-tv) \exp\left(-\int_0^t a(s,x+(s-t)v)ds\right) \\
&\quad + \mathbf{1}_{t > \tau_x} f_b^- (t-\tau_x,x-\tau_x v) \exp\left(-\int_{t-\tau_x}^t a(s,x+(s-t)v)ds\right) \\
&\quad + \int_{t-\tau_x}^t \exp\left(-\int_{s}^{t}a(\tau,x+(\tau-t)v)d\tau\right) S(s,x+(s-t)v)ds
\end{aligned}}
\]


\subsection{Solutions généralisées}
Dans cette section $\Omega$ n'est plus forcément convexe. Ceci à pour conséquence que rien n'assure la continuités des éventuelles solutions, pourtant en pratique ces cas doivent être traités. On défini ci-dessous la notion de solution généralisés qui permet de traiter avec ce problème en évitant d'utiliser la théorie des distributions.


\paragraph{Définition}

Soient $\Omega$ un ouvert à bord de classe $\mathcal{C}^1$ de $\mathbb{R}^n$, $a$ et $S$ continues sur $\mathbb{R}_+ \times \Omega$. On dit que $f$ est solution généralisé du problème 

$$\frac{\partial f}{\partial t}(t,x)+v \cdot \nabla_x f(t,x)+ a(t,x)f(t,x)=S(t,x), \quad x \in \mathbb{R}^N, \quad t>0.\\$$

si et seulement si la fonction 

$s \mapsto f(t+s,x+sv)$ est de classe $\mathcal{C}^1 ~ p.p.~en~(t,x) \in \mathbb{R}_+ \times \Omega$ et vérifie

$\frac{d}{ds}f(t+s,x+sv) +a(t+s,x+sv)f(t+s,x+sv) = S(t+s,x+sv) \quad (t+s,x+sv) \in \mathbb{R}_+^* \times \Omega$.


\paragraph{}



Soient $\Omega$ un ouvert à bord de classe $\mathcal{C}^1$ de $\mathbb{R}^n$, $a$ et $S$ continues sur $\mathbb{R}_+ \times \Omega$. Soient  $f_b^- \in \mathbf{L}_{loc}^{\infty}(\mathbb{R}_+ \times \partial \Omega^-)$, $f^{in} \in \mathbf{L}_{loc}^{\infty}(\overline{\Omega})$

Dans ce cas le problème :

\[
\begin{cases}
\frac{\partial f}{\partial t}(t,x)+v \cdot \nabla_x f(t,x)+ a(t,x)f(t,x)=S(t,x), \quad x \in \mathbb{R}^N, \quad t>0.\\
\restriction{f}{\partial \Omega}= f_b^-\\
\restriction{f}{t=0}= f^{in}(x)
\end{cases}
\]

admet une unique solution généralisée. La formule de cette solution est la même que dans le cas précédent $i.e.~ p.p.~en~(t,x) \in \mathbb{R}_+ \times \Omega$ on a:

\[
\boxed{
\begin{aligned}
f(t,x) &= \mathbf{1}_{t \leq \tau_x} f^{in} (x-tv) \exp\left(-\int_0^t a(s,x+(s-t)v)ds\right) \\
&\quad + \mathbf{1}_{t > \tau_x} f_b^- (t-\tau_x,x-\tau_x v) \exp\left(-\int_{t-\tau_x}^t a(s,x+(s-t)v)ds\right) \\
&\quad + \int_{t-\tau_x}^t \exp\left(-\int_{s}^{t}a(\tau,x+(\tau-t)v)d\tau\right) S(s,x+(s-t)v)ds
\end{aligned}}
\]







\section{\textcolor{blue}{Equation de Boltzmann linéaire}}

Dans cette section nous allons nous concentrer sur la véritable équation de Boltzmann. Cette équation intégro-différentielle est décrite en \ref{Boltzmann}.

\begin{equation} \label{Boltzmann}
\frac{\partial f}{\partial t}(t,x,v)+v \cdot \nabla_x f(t,x,v) + a(t,x)f(t,x,v) = \int_{\mathbb{R}^n}k(t,x,v,w)f(t,x,w)d\mu(w)+Q(t,x,v).
\end{equation}

Pour simplifier les notations cette équation sera écrite comme : $\frac{\partial f}{\partial t}+v \cdot \nabla_x f +af =   \mathcal{K} f +Q $

Ici $v$ et $k$ sont des fonctions continues données. Le champ des possibles pour la vitesse $v$ est très varié, du fait de la diversité des problèmes physiques qui peuvent être considérés. Par conséquent la notation $d\mu(w)$ englobe toutes ces possibilités en ayant une fonction de poids. La majeur différence avec l'équation de transport est que cette fois-ci nous considérons les phénomènes d'échanges et donc $v$ ne peut plus être vu comme un simple paramètre.
Dans cette section nous ne pouvons pas mettre en évidence l'expression exacte des solutions car la méthode des caractéristiques donne une autre équaton intégrale faisant intervenir la solution.




\paragraph{Définition}

Soient $Q$ continue sur $]0,T[\times \Omega \times \mathbb{R}^n$. On dit que $f$ continue sur $]0,T[\times \Omega \times \mathbb{R}^n$ est solution généralisé du problème 


$$\frac{\partial f}{\partial t}+v \cdot \nabla_x f +af = \mathcal{K} f +Q\\$$

\paragraph{}
si et seulement si $\forall (t,x,v) \in ]0,T[\times \Omega \times \mathbb{R}^n$ la fonction 

$s \mapsto f(t+s,x+sv,v)$ est de classe $\mathcal{C}^1 ~\forall x+sv \in  \Omega$ et vérifie $\forall s \in \mathbb{R} ~ t.q.~ x +sv \in \Omega$

$\frac{d}{ds}f(t+s,x+sv,v) +a(t+s,x+sv,v)f(t+s,x+sv,v) = (\mathcal{K}f+ Q)(t+s,x+sv,v).$

\subsection{Problème de Cauchy}
\paragraph{}



Soient $f^{in} \in \mathbf{C}_b(\mathbb{R}^n \times \mathbb{R}^n)$ et $Q \in \mathbf{C}_b(]0,T[ \times \mathbb{R}^n \times \mathbb{R}^n)$.
Soient $0 \leq a \in \mathbf{C}_b(]0,T[ \times \mathbb{R}_x^n \times \mathbb{R}_v^n)$ et $0 \leq k \in \mathbf{C}_b(]0,T[ \times \mathbb{R}_x^n \times \mathbb{R}_v^n \times \mathbb{R}_w^n)$. La notation $\mathbf{C}_b(X)$ désigne l'ensemble des fonctions continues bornées sur $X$ à valeurs dans $\mathbb{R}$\\


On suppose que :  $\underset{(t,x,v) \in ]0,T[ \times \mathbb{R}^n \times \mathbb{R}^n}{\sup} \int_{\mathbb{R}^n} k(t,x,v,w)d\mu(w) < +\infty$

\paragraph{}

Alors le problème 

\[
\begin{cases}
\frac{\partial f}{\partial t}+v \cdot \nabla_x f +af =   \mathcal{K} f +Q \quad (t,x,v) \in ]0,T[ \times \mathbb{R}^n \times \mathbb{R}^n\\
\restriction{f}{t=0}= f^{in}
\end{cases}
\]
Admet une unique solution généralisée $f \in \mathbf{C}_b(]0,T[ \times \mathbb{R}^n \times \mathbb{R}^n)$.

\subsection{Problème au limite}

On définit l'ensemble $\Gamma_-$ comme étant $\Gamma_-= \{(x,v) \in \partial \Omega \times \mathcal{R}^n ~ | ~ v\cdot n_x<0\}$

Soient $f^{in} \in \mathbf{C}_b(\overline{\Omega} \times \mathbb{R}^n)$, $f_b^- \in \mathbf{C}_b([0,T] \times \Gamma_-)$ telle que $f_b^-(0,y,v)=f^{in}(y,v) \quad \forall (y,v) \in \Gamma_-$

Soit $Q \in \mathbf{C}_b([0,T] \times \overline{\Omega} \times \mathbb{R}^n)$.

Soient $0 \leq a \in \mathbf{C}_b(\mathbb{R}_+ \times \Omega_x \times \mathbb{R}_v^n)$ et $0 \leq k \in \mathbf{C}_b(\mathbb{R}_+ \times \Omega_x^n \times \mathbb{R}_v^n \times \mathbb{R}_w^n)$.

Alors le problème 

\[
\begin{cases}
\frac{\partial f}{\partial t}+v \cdot \nabla_x f +af =   \mathcal{K} f +Q \quad (t,x,v) \in ]0,T[ \times \Omega \times \mathbb{R}^n\\
\restriction{f}{t=0}= f^{in}\\
\restriction{f}{\Gamma^-}=f_b^-
\end{cases}
\]

admet une unique solution généralisée $f \in \mathbf{C}_b([0,T] \times \Omega \times \mathbb{R}^n)$.




\end{document}

