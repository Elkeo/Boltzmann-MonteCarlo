\documentclass[a4paper,12pt]{article}
\usepackage[left=2cm,right=2cm,top=2cm,bottom=2cm]{geometry}
\usepackage{xcolor}
\usepackage{graphicx}
\usepackage{tikz}
\bibliographystyle{unsrt}
\usepackage{float}
\usepackage{amsmath}
\usepackage{siunitx}
\usepackage{subcaption}
\usepackage{fancyhdr}
\usepackage[utf8]{inputenc}
\usepackage{verbatim}
\usepackage[T1]{fontenc}
\usepackage{hyperref}
\usepackage{enumitem}
\usepackage{fontawesome5} % Pour les icônes (utilisées pour les logos)

\definecolor{titleblue}{RGB}{0,38,76} % Couleur bleue pour le titre
\definecolor{lineblue}{RGB}{0,28,56}  % Couleur bleue pour les lignes
\definecolor{darkgray}{RGB}{80,80,80} % Couleur grise pour le texte
\definecolor{lightblue}{RGB}{100,150,190}

\begin{document}
	
	\subsection{The use of Monte Carlo methods to solve the linear Boltzmann equation.}
	\textbf The transport equation in an infinite medium with its corresponding deterministic collisional component can be expressed as:
	\begin{equation}
		\partial _t u(x,t,\textbf{v}) + \textbf{v} \cdot \nabla u(x,t,\textbf{v}) + v\sigma_t (x,t,\textbf{v})u(x,t,\textbf{v})= v\sigma_s(x,t,\textbf{v})\int P (x,t,\textbf{v},\textbf{v}')u(x,t,\textbf{v}')d\textbf{v}'
	\end{equation}
	
	Where 
	\begin{equation*}
		\sigma_s (x,t,\textbf{v})= \int \sigma_s (x,t,\textbf{v},\textbf{v}')d\textbf{v}', \quad  P (x,t,\textbf{v},\textbf{v}')=
		\frac{\sigma_s (x,t,\textbf{v},\textbf{v}')}{\sigma_s (x,t,\textbf{v})}
	\end{equation*}
	
	
	
	
The approach involves a series of variable changes. The initial step involves re-expressing the transport equation \ref{ref11} with respect to a characteristic $x + vt$. As a result, it transforms into:

\begin{equation}
	\partial _s u(x+\textbf{v}s,s,\textbf{v}) = -v\sigma_t (x+\textbf{v}s,s,\textbf{v})u(x+\textbf{v}s,s,\textbf{v}) + v\sigma_s(x+\textbf{v}s,s,\textbf{v})\int P (x+\textbf{v}s,s,\textbf{v},\textbf{v}')u(x+\textbf{v}s,s,\textbf{v}')d\textbf{v}'
\end{equation}

Let's multiply both sides of the equation by:
\begin{equation*}
	e^{\int _0^s v\sigma_t (x + \textbf{v}\alpha,\alpha, v) d\alpha}
\end{equation*}
Following that, we obtain
\begin{equation*}
	\partial _s [u(x+\textbf{v}s,s,\textbf{v})e^{\int _0^s v\sigma_t (x + \textbf{v}\alpha,\alpha, v) d\alpha}] = e^{\int _0^s v\sigma_t (x + \textbf{v}\alpha,\alpha, v) d\alpha} v\sigma_s(x+\textbf{v}s,s,\textbf{v})\int P (x+\textbf{v}s,s,\textbf{v},\textbf{v}')u(x+\textbf{v}s,s,\textbf{v}')d\textbf{v}' \label{ref12}
\end{equation*}
We get after integrating the equation \ref{ref12} between [0, t]:

\begin{multline}
	u(x+\textbf{v}t,t,\textbf{v}) = u_0(x, \textbf{v}) \exp\left(- \int_{0}^{t} v\sigma_t\left(x + \textbf{v} \alpha, \alpha, \textbf{v}\right) d\alpha\right) \\
	+ \int_{0}^{t} \int v\sigma_s\left(x + \textbf{v}s, s, \textbf{v}\right) u\left(x + \textbf{v}s, s, \textbf{v}'\right) e^{- \int_s^t v\sigma_t\left(x + \textbf{v} \alpha, \textbf{v}\right) d\alpha} P\left(x + \textbf{v} s, s, \textbf{v}, \textbf{v}'\right) d\textbf{v}'ds 
\end{multline}

We obtain:

\begin{multline}
	u(x,t,\textbf{v}) = u_0(x - \textbf{v}t, \textbf{v}) \exp\left(- \int_{0}^{t} v\sigma_t\left(x - \textbf{v}(t - \alpha), \alpha, \textbf{v}\right) d\alpha\right) \\
	+ \int_{0}^{t} \int v\sigma_s\left(x - \textbf{v}(t - s), s, \textbf{v}\right) u\left(x - \textbf{v}(t - s), s, \textbf{v}'\right) e^{- \int_s^t v\sigma_t\left(x - \textbf{v}(t - \alpha), \textbf{v}\right) d\alpha} P\left(x - \textbf{v}(t - s), s, \textbf{v}, \textbf{v}'\right) d\textbf{v}'ds \label{ref1}
\end{multline}
We also have:
\begin{multline}
	\exp\left(- \int_{0}^{t} v\sigma_t\left(x - \textbf{v}(t - \alpha), \alpha, \textbf{v}\right) d\alpha\right) = \exp\left(- \int_{0}^{t} v\sigma_t\left(x - \textbf{v} \alpha,t- \alpha, \textbf{v}\right) d\alpha\right) \\ = \int _t^\infty  v\sigma_t\left(x - \textbf{v} s,t- s, \textbf{v}\right)
	\exp\left(- \int_{0}^{s} v\sigma_t\left(x - \textbf{v} \alpha,t- \alpha, \textbf{v}\right) d\alpha\right) ds
\end{multline}
Then the integral representation of \ref{ref11} is provided by:

\begin{multline}
	u(x,t,\textbf{v}) =  \int _t^\infty  u_0(x - \textbf{v}t, \textbf{v}) v\sigma_t\left(x - \textbf{v} s,t- s, \textbf{v}\right)
	\exp\left(- \int_{0}^{s} v\sigma_t\left(x - \textbf{v} \alpha,t- \alpha, \textbf{v}\right) d\alpha\right) ds\\
	+ \int_{0}^{t} \int v\sigma_s\left(x - \textbf{v}(t - s), s, \textbf{v}\right) u\left(x - \textbf{v}(t - s), s, \textbf{v}'\right) e^{- \int_s^t v\sigma_t\left(x - \textbf{v}(t - \alpha), \textbf{v}\right) d\alpha} P\left(x - \textbf{v}(t - s), s, \textbf{v}, \textbf{v}'\right) d\textbf{v}'ds \label{ref13}
\end{multline}

Developing a Monte Carlo scheme involves introducing a group of random variables along with their associated probability measure to express \ref{ref13} as an expectation. The selection of this set of random variables is not unique, resulting in various Monte Carlo schemes with distinct properties.
	
	
\end{document}
