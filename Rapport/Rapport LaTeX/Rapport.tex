\documentclass[a4paper, 11pt]{article}
\usepackage{inputenc}
\usepackage{amsmath}
\usepackage{graphicx}
\usepackage[T1]{fontenc}
\usepackage{babel}
\usepackage{hyperref}    %pour indexer la table des matières 
\hypersetup{pdfborder=1 1 1}
\usepackage{float} 
\usepackage{url}
\usepackage{caption}
\usepackage{multirow}
\usepackage{multicol}
\usepackage{listings}
\usepackage{subcaption}
\usepackage{empheq}
\usepackage{fancyhdr}
\usepackage{booktabs}
\usepackage{amssymb}
\usepackage{xcolor}
\usepackage{setspace}
\usepackage{enumitem}
\setlength{\hoffset}{-18pt}         
\setlength{\oddsidemargin}{0pt} % Marge gauche sur pages impaires
\setlength{\evensidemargin}{9pt} % Marge gauche sur pages paires
\setlength{\marginparwidth}{54pt} % Largeur de note dans la marge
\setlength{\textwidth}{481pt} % Largeur de la zone de texte (17cm)
\setlength{\voffset}{-18pt} % Bon pour DOS
\setlength{\marginparsep}{7pt} % Séparation de la marge
\setlength{\topmargin}{0pt} % Pas de marge en haut
\setlength{\headheight}{13pt} % Haut de page
\setlength{\headsep}{10pt} % Entre le haut de page et le texte
\setlength{\footskip}{27pt} % Bas de page + séparation
\setlength{\textheight}{708pt} % Hauteur de la zone de texte (25cm)
%Traits en haut et en bas des pages
\usepackage{fancyhdr}
\pagestyle{fancy}
\fancyhead[L]{}
\fancyhead[R]{}
\renewcommand\footrulewidth{1pt}
\renewcommand\footrulewidth{1pt}
\fancyfoot[L]{\tiny Project Report}
\usepackage{lastpage}
\fancyfoot[R]{\tiny January 2024}
\fancyfoot[C]{\textbf{Page \thepage/\pageref{LastPage}}}
\lstnewenvironment{mathematicacode}[1][]
{
	\lstset{
		language=Mathematica,
		basicstyle=\small\ttfamily,
		numbers=left,
		numberstyle=\tiny,
		numbersep=5pt,
		frame=single,
		frameround=tttt,
		framexleftmargin=5pt,
		#1
	}
}
{}

\begin{document} 
	
	%----------------------------------------------------------------------------------------
	%	TITLE PAGE
	%----------------------------------------------------------------------------------------
	
	\begin{titlepage} % Suppresses displaying the page number on the title page and the subsequent page counts as page 1
		\newcommand{\HRule}{\rule{\linewidth}{0.5mm}} % Defines a new command for horizontal lines, change thickness here
		
		\centering % Centre everything on the page
		
		%------------------------------------------------
		%	Headings
		%------------------------------------------------
		
		\textsc{\LARGE ENSEIRB-MATMECA}\\[1cm] % Main heading such as the name of your university/college
		
		
		\begin{figure}[H]
			\begin{center}
				\includegraphics[scale=4]{logo ecole.jpg}
			\end{center}
		\end{figure}
		
		%\textsc{\large Minor Heading}\\[0.5cm] % Minor heading such as course title
		
		%------------------------------------------------
		%	Title
		%------------------------------------------------
		
		\vspace{1cm}
		
		\HRule\\[0.4cm]
		
		{\huge\bfseries Solving the linear Boltzmann equation using Monte Carlo methods }\\[0.4cm] % Title of your document
		
		\HRule\\[1.5cm]
		
		{\huge\bfseries }
		
		\bigskip
		\bigskip
		
		\centering
		\Large{Project lead by :} \\
		\Large{C. Aumonier, A. Boucher, G. Doyen, K. El Maddah, G. Rodiere}
		
		%------------------------------------------------
		%	Date
		%------------------------------------------------
		
		\vfill\vfill\vfill % Position the date 3/4 down the remaining page
		{\Large  Project supervised by :}\\
		\Large{G. Poëtte}
		
		\vspace{0,5cm}
		
		{\large January 2024} % Date, change the \today to a set date if you want to be precise
		
		%------------------------------------------------
		%	Logo
		%------------------------------------------------
		
		%\vfill\vfill
		%\includegraphics[width=0.2\textwidth]{placeholder.jpg}\\[1cm] % Include a department/university logo - this will require the graphicx package
		
		%----------------------------------------------------------------------------------------
		
		\vfill % Push the date up 1/4 of the remaining page
		
	\end{titlepage}
	
\tableofcontents

\newpage
	
\section{Introduction}

**Gabriel

\section{Transport equations}

insérer la partie écrite par Clément
**Gabriel traduction

\section{The use of Monte Carlo methods to solve the linear Boltzmann equation}

The transport equation in an infinite medium with its corresponding deterministic collisional component can be expressed as:
\begin{equation}
	\partial _t u(x,t,\textbf{v}) + \textbf{v} \cdot \nabla u(x,t,\textbf{v}) + v\sigma_t (x,t,\textbf{v})u(x,t,\textbf{v})= v\sigma_s(x,t,\textbf{v})\int P (x,t,\textbf{v},\textbf{v}')u(x,t,\textbf{v}')d\textbf{v}' \label{ref11}
\end{equation}

Where 
\begin{equation*}
	\sigma_s (x,t,\textbf{v})= \int \sigma_s (x,t,\textbf{v},\textbf{v}')d\textbf{v}', \quad  P (x,t,\textbf{v},\textbf{v}')=
	\frac{\sigma_s (x,t,\textbf{v},\textbf{v}')}{\sigma_s (x,t,\textbf{v})}
\end{equation*}

The approach involves a series of variable changes. The initial step involves re-expressing the transport equation \ref{ref11} with respect to a characteristic $x + vt$. As a result, it transforms into:

\begin{equation}
	\partial _s u(x+\textbf{v}s,s,\textbf{v}) = -v\sigma_t (x+\textbf{v}s,s,\textbf{v})u(x+\textbf{v}s,s,\textbf{v}) + v\sigma_s(x+\textbf{v}s,s,\textbf{v})\int P (x+\textbf{v}s,s,\textbf{v},\textbf{v}')u(x+\textbf{v}s,s,\textbf{v}')d\textbf{v}'
\end{equation}

Let's multiply both sides of the equation by:
\begin{equation*}
	e^{\int _0^s v\sigma_t (x + \textbf{v}\alpha,\alpha, v) d\alpha}
\end{equation*}
Following that, we obtain
\begin{equation*}
	\partial _s [u(x+\textbf{v}s,s,\textbf{v})e^{\int _0^s v\sigma_t (x + \textbf{v}\alpha,\alpha, v) d\alpha}] = e^{\int _0^s v\sigma_t (x + \textbf{v}\alpha,\alpha, v) d\alpha} v\sigma_s(x+\textbf{v}s,s,\textbf{v})\int P (x+\textbf{v}s,s,\textbf{v},\textbf{v}')u(x+\textbf{v}s,s,\textbf{v}')d\textbf{v}' \label{ref12}
\end{equation*}
We get after integrating the equation \ref{ref12} between [0, t]:

\begin{multline}
	u(x+\textbf{v}t,t,\textbf{v}) = u_0(x, \textbf{v}) \exp\left(- \int_{0}^{t} v\sigma_t\left(x + \textbf{v} \alpha, \alpha, \textbf{v}\right) d\alpha\right) \\
	+ \int_{0}^{t} \int v\sigma_s\left(x + \textbf{v}s, s, \textbf{v}\right) u\left(x + \textbf{v}s, s, \textbf{v}'\right) e^{- \int_s^t v\sigma_t\left(x + \textbf{v} \alpha, \textbf{v}\right) d\alpha} P\left(x + \textbf{v} s, s, \textbf{v}, \textbf{v}'\right) d\textbf{v}'ds 
\end{multline}

We obtain:

\begin{multline}
	u(x,t,\textbf{v}) = u_0(x - \textbf{v}t, \textbf{v}) \exp\left(- \int_{0}^{t} v\sigma_t\left(x - \textbf{v}(t - \alpha), \alpha, \textbf{v}\right) d\alpha\right) \\
	+ \int_{0}^{t} \int v\sigma_s\left(x - \textbf{v}(t - s), s, \textbf{v}\right) u\left(x - \textbf{v}(t - s), s, \textbf{v}'\right) e^{- \int_s^t v\sigma_t\left(x - \textbf{v}(t - \alpha), \textbf{v}\right) d\alpha} P\left(x - \textbf{v}(t - s), s, \textbf{v}, \textbf{v}'\right) d\textbf{v}'ds \label{ref1}
\end{multline}
We also have:
\begin{multline}
	\exp\left(- \int_{0}^{t} v\sigma_t\left(x - \textbf{v}(t - \alpha), \alpha, \textbf{v}\right) d\alpha\right) = \exp\left(- \int_{0}^{t} v\sigma_t\left(x - \textbf{v} \alpha,t- \alpha, \textbf{v}\right) d\alpha\right) \\ = \int _t^\infty  v\sigma_t\left(x - \textbf{v} s,t- s, \textbf{v}\right)
	\exp\left(- \int_{0}^{s} v\sigma_t\left(x - \textbf{v} \alpha,t- \alpha, \textbf{v}\right) d\alpha\right) ds
\end{multline}
Then the integral representation of \ref{ref11} is provided by:

\begin{multline}
	u(x,t,\textbf{v}) =  \int _t^\infty  u_0(x - \textbf{v}t, \textbf{v}) v\sigma_t\left(x - \textbf{v} s,t- s, \textbf{v}\right)
	\exp\left(- \int_{0}^{s} v\sigma_t\left(x - \textbf{v} \alpha,t- \alpha, \textbf{v}\right) d\alpha\right) ds\\
	+ \int_{0}^{t} \int v\sigma_s\left(x - \textbf{v}(t - s), s, \textbf{v}\right) u\left(x - \textbf{v}(t - s), s, \textbf{v}'\right) e^{- \int_s^t v\sigma_t\left(x - \textbf{v}(t - \alpha), \textbf{v}\right) d\alpha} P\left(x - \textbf{v}(t - s), s, \textbf{v}, \textbf{v}'\right) d\textbf{v}'ds \label{ref13}
\end{multline}


\section{Semi analog MC scheme}

Developing a Monte Carlo scheme involves introducing a group of random variables along with their associated probability measure to express \ref{ref13} as an expectation. The selection of this set of random variables is not unique, resulting in various Monte Carlo schemes with distinct properties.

\section{Numerical tests/results}

à voir

\section{Conclusion}


	
	
\end{document}