\documentclass[a4paper, 11pt]{article}
\usepackage{inputenc}
\usepackage{amsmath}
\usepackage{graphicx}
\usepackage[T1]{fontenc}
\usepackage{babel}
\usepackage{hyperref}    %pour indexer la table des matières 
\hypersetup{pdfborder=1 1 1}
\usepackage{float} 
\usepackage{url}
\usepackage{caption}
\usepackage{multirow}
\usepackage{multicol}
\usepackage{listings}
\usepackage{subcaption}
\usepackage{empheq}
\usepackage{fancyhdr}
\usepackage{booktabs}
\usepackage{amssymb}
\usepackage{xcolor}
\usepackage{setspace}
\usepackage{amsmath} % pour les notations mathématiques
\usepackage{algorithm} % pour l'environnement algorithm
\usepackage{algpseudocode} % pour l'environnement algpseudocode
\usepackage{enumitem}
\setlength{\hoffset}{-18pt}         
\setlength{\oddsidemargin}{0pt} % Marge gauche sur pages impaires
\setlength{\evensidemargin}{9pt} % Marge gauche sur pages paires
\setlength{\marginparwidth}{54pt} % Largeur de note dans la marge
\setlength{\textwidth}{481pt} % Largeur de la zone de texte (17cm)
\setlength{\voffset}{-18pt} % Bon pour DOS
\setlength{\marginparsep}{7pt} % Séparation de la marge
\setlength{\topmargin}{0pt} % Pas de marge en haut
\setlength{\headheight}{13pt} % Haut de page
\setlength{\headsep}{10pt} % Entre le haut de page et le texte
\setlength{\footskip}{27pt} % Bas de page + séparation
\setlength{\textheight}{708pt} % Hauteur de la zone de texte (25cm)
%Traits en haut et en bas des pages
\usepackage{fancyhdr}
\pagestyle{fancy}
\fancyhead[L]{}
\fancyhead[R]{}
\renewcommand\footrulewidth{1pt}
\renewcommand\footrulewidth{1pt}
\fancyfoot[L]{\tiny Project Report}
\usepackage{lastpage}
\fancyfoot[R]{\tiny January 2024}
\fancyfoot[C]{\textbf{Page \thepage/\pageref{LastPage}}}
\lstnewenvironment{mathematicacode}[1][]
{
	\lstset{
		language=Mathematica,
		basicstyle=\small\ttfamily,
		numbers=left,
		numberstyle=\tiny,
		numbersep=5pt,
		frame=single,
		frameround=tttt,
		framexleftmargin=5pt,
		#1
	}
}
{}

\def\restriction#1#2{\mathchoice
	{\setbox1\hbox{${\displaystyle #1}_{\scriptstyle #2}$}
		\restrictionaux{#1}{#2}}
	{\setbox1\hbox{${\textstyle #1}_{\scriptstyle #2}$}
		\restrictionaux{#1}{#2}}
	{\setbox1\hbox{${\scriptstyle #1}_{\scriptscriptstyle #2}$}
		\restrictionaux{#1}{#2}}
	{\setbox1\hbox{${\scriptscriptstyle #1}_{\scriptscriptstyle #2}$}
		\restrictionaux{#1}{#2}}}
\def\restrictionaux#1#2{{#1\,\smash{\vrule height .8\ht1 depth .85\dp1}}_{\,#2}} 

\begin{document} 
	
	%----------------------------------------------------------------------------------------
	%	TITLE PAGE
	%----------------------------------------------------------------------------------------
	
	\begin{titlepage} % Suppresses displaying the page number on the title page and the subsequent page counts as page 1
		\newcommand{\HRule}{\rule{\linewidth}{0.5mm}} % Defines a new command for horizontal lines, change thickness here
		
		\centering % Centre everything on the page
		
		%------------------------------------------------
		%	Headings
		%------------------------------------------------
		
		\textsc{\LARGE ENSEIRB-MATMECA}\\[1cm] % Main heading such as the name of your university/college
		
		
		\begin{figure}[H]
			\begin{center}
				\includegraphics[scale=4]{logo ecole.jpg}
			\end{center}
		\end{figure}
		
		%\textsc{\large Minor Heading}\\[0.5cm] % Minor heading such as course title
		
		%------------------------------------------------
		%	Title
		%------------------------------------------------
		
		\vspace{1cm}
		
		\HRule\\[0.4cm]
		
		{\huge\bfseries Solving the linear Boltzmann equation using Monte Carlo methods }\\[0.4cm] % Title of your document
		
		\HRule\\[1.5cm]
		
		{\huge\bfseries }
		
		\bigskip
		\bigskip
		
		\centering
		\Large{Project lead by :} \\
		\Large{C. Aumonier, A. Boucher, G. Doyen, K. El Maddah, G. Rodiere}
		
		%------------------------------------------------
		%	Date
		%------------------------------------------------
		
		\vfill\vfill\vfill % Position the date 3/4 down the remaining page
		{\Large  Project supervised by :}\\
		\Large{G. Poëtte}
		
		\vspace{0,5cm}
		
		{\large January 2024} % Date, change the \today to a set date if you want to be precise
		
		%------------------------------------------------
		%	Logo
		%------------------------------------------------
		
		%\vfill\vfill
		%\includegraphics[width=0.2\textwidth]{placeholder.jpg}\\[1cm] % Include a department/university logo - this will require the graphicx package
		
		%----------------------------------------------------------------------------------------
		
		\vfill % Push the date up 1/4 of the remaining page
		
	\end{titlepage}
	
\tableofcontents

\newpage
	
\section{Introduction}


	
\subsection{Terms and definitions}
	
	The transport equation plays a crucial role in describing the behavior of particles in various physical systems. One of its important derivatives, the linear Boltzmann equation, is governed by the evolution of particle distribution in phase space. In this project, a numerical solution for the linear Boltzmann equation is developed utilizing the Monte Carlo method. The report is structured as follows:
	
	\begin{enumerate}
		\item \textbf{Transport Equation:} The fundamental transport equation is introduced, providing a basis for understanding the derivation of the linear Boltzmann equation.
		
		\item \textbf{Linear Boltzmann Equation:} Building upon the transport equation, the linear Boltzmann equation is explored, emphasizing its significance in characterizing particle transport phenomena.
		
		\item \textbf{Monte Carlo Method for Linear Boltzmann Equation:} The approach to solving the linear Boltzmann equation using the Monte Carlo method is presented, along with a discussion of the advantages and challenges associated with this numerical technique.
		
		\item \textbf{Semi-Analog Monte Carlo Scheme:} A semi-analog Monte Carlo scheme is introduced as an innovative strategy to enhance the efficiency and accuracy of simulations.
	\end{enumerate}
	
\subsection{Context and Applications}
	
	The linear Boltzmann equation finds applications in a wide range of industrial and research contexts. Notable examples include its use in nuclear reactor physics, radiation shielding design, medical imaging, and astrophysics. Applications extend to industries such as nuclear energy, healthcare, and aerospace, where accurate simulations of particle transport are crucial for optimizing designs and ensuring safety protocols.
	
	In the realm of academic research, the linear Boltzmann equation serves as a cornerstone in the study of rarefied gas dynamics, plasma physics, and neutron transport. Collaborations between universities, research institutions, and companies contribute to advancements in understanding and solving complex problems associated with particle transport phenomena.
	
	
	\medbreak
	
	This project aims to contribute to the broader understanding and application of the linear Boltzmann equation, with a focus on developing efficient numerical methods for solving this intricate problem.


\paragraph{Note} Here is a brief summary of the various existences and uniqueness of solutions based on different cases of equations to be solved. The primary source used is the transport and diffusion document.
\section{mathematical analysis}

\subsection{Transport Equation}

\paragraph{}
In this section, we will examine the solutions of the transport equation, a particular case of the Boltzmann equation, defined below:



Let's $v \in \mathbb{R}^N$

\begin{equation} \label{transport}
	\frac{\partial f}{\partial t}(t,x)+v \cdot \nabla_x f(t,x) + a(t,x)f(t,x) = S(t,x), \quad x \in \mathbb{R}^N, \quad t>0.
\end{equation}


$f$ is the unknown function, $a$ is the given damping or amplification rate, and $S$ is the given source term.

\subsubsection{Cauchy Problem}
\paragraph{Special Case without Absorption or Source Term}

Let's consider the following Cauchy problem:

\[
\begin{cases}
	\frac{\partial f}{\partial t}(t,x)+v \cdot \nabla_x f(t,x)=0, \quad x \in \mathbb{R}^N, \quad t>0.\\
	f(0,x)) = f^{in}(x)
\end{cases}
\]

If $f^{in} \in \mathcal{C}^1(\mathbb{R}^n)$, then using the method of characteristics, we can easily demonstrate that the Cauchy problem has a unique solution $f \in \mathcal{C}^1(\mathbb{R}_+ \times \mathbb{R}^n)$. This solution is given as follows:





$\boxed{f(t,x)=f^{in}(x-tv) \quad \forall (t,x) \in \mathbb{R}_+ \times \mathbb{R}^n }$




\paragraph{With Absorption and Source Term}



This time, we consider damping and the source term, assuming they belong to $\mathcal{C}^1(\mathbb{R}_+ \times \mathbb{R}^n)$. The considered Cauchy problem is as follows:


\[
\begin{cases}
\frac{\partial f}{\partial t}(t,x)+v \cdot \nabla_x f(t,x) + a(t,x)f(t,x) = S(t,x), \quad x \in \mathbb{R}^N, \quad t>0.\\
f(0,x)) = f^{in}(x)
\end{cases}
\]

In this case, if $f^{in} \in \mathcal{C}^1(\mathbb{R}^n)$, then using the method of characteristics, we can still demonstrate that the Cauchy problem has a unique solution $f \in \mathcal{C}^1(\mathbb{R}_+ \times \mathbb{R}^n)$. This solution is given as follows:\\
$\boxed{f(t,x)=f^{in}(x-tv) e^{-\int_0 ^t a(\tau,x+(t-\tau)v)d\tau} + \int_0 ^t  e^{-\int_0 ^t a(\tau,x+(t-\tau)v)d\tau} S(s,x+(s-t)v)ds \quad \forall (t,x) \in \mathbb{R}_+ \times \mathbb{R}^n }$

We will demonstrate the previous formula using the characteristics method. We will not demonstrate all folowing expression of the exact solution but the idea would be the same.

\paragraph{}

We will demonstrate the previous formula using the method of characteristics. The proof of the following expression for the exact solution will not be elaborated upon, but the underlying idea would be analogous. The interested reader is invited to see \textcolor{red}{Mettre la source allaire!!!!!}.

Let $y \in \mathbb{R}$, and consider the characteristic curve emanating from $y$ at $t=0$, given by :

\[ \{(t,\gamma(t)), | t\geq 0\} , \text{ where } \gamma (t) 	y+tv\]

If $f \in \mathcal{C}^1(\mathbb{R}_+ \times \mathbb{R}^n$ is the solution of the previous Cauchy problem, then $t \mapsto f(t,\gamma(t))$ is of $\mathcal{C}^1$ class on $\mathbb{R}_+$, and 


\[
\begin{aligned}
\frac{d f}{d t}(t,\gamma(t))&=(\frac{\partial f}{\partial t}
v \cdot \nabla_x f(t,x))(t,\gamma(t)) \\ &= S(t,\gamma(t))-a(t,\gamma(t))f(t,\gamma(t)), \quad t>0.
\end{aligned}
\]

The previous partial differential equation reduces to an ordinary differential equation (ODE) of the form:



\[
\begin{cases}
u'(t)+\alpha(t)u(t)=\Sigma(t), \quad t>0,\\
u(0)=u^{in}
\end{cases}
\]

The well-known solution is calculated using the constant variation method:

\[ u(t) = u^{in} e^{-A(t)} + \int_{0}^{t} e^{-A(t)-A(s)}\Sigma(s)ds, \text{ with } A(t)=\int_{0}^{t} \alpha (\tau) d\tau. \]

Then by applying this to the Cauchy problem along the characteristic curve from $y$ at $t=0$, we obtain:

\[ f(t,y+tv) = f^{in}(y) e^{\int_{0}^{t}a(\tau,y+\tau v)d\tau} + \int_{0}^{t} e^{\int_{0}^{t}a(\tau,y+\tau v)d\tau}S(s,y+sv)ds, \quad y\in \mathbb{R}^n, t \geq 0.\]

By replacing the dummy variable $y$ with $x-tv$ we arrive the expected formula.

\subsubsection{Boundary Value Problems}

In the previous section, we presented results applicable when considering complete freedom for the spatial variable in $\mathbb{R}^n$. However, for a significant portion of physical problems, this may not be the case. That is why we are more interested in a problem of the boundary value type. In this section, we will consider the set $\Omega$, which is an open set with a $\mathcal{C}^1$ boundary in $\mathbb{R}^n$. Additionally, we take $v \in \mathbb{R}^n\backslash \{0\}$.

\paragraph{Special Case without Absorption or Source Term}

\paragraph{}
If, moreover, $\Omega$ is convex, and let $f_b^- \in \mathcal{C}^1(\mathbb{R}_+ \times \partial \Omega^-)$ and $f^{in} \in \mathcal{C}^1(\overline{\Omega})$, where $\partial \Omega^- = \{x\in\partial \Omega~ | ~ v\cdot n_x<0\}$, then the problem:


\[
\begin{cases}
	\frac{\partial f}{\partial t}(t,x)+v \cdot \nabla_x f(t,x)=0, \quad x \in \mathbb{R}^N, \quad t>0.\\
	\restriction{f}{\partial \Omega}= f_b^-\\
	\restriction{f}{t=0}= f^{in}(x)
\end{cases}
\]

has a unique solution $f \in \mathcal{C}^1(\mathbb{R}_+ \times \Omega)$ and $f \in \mathcal{C}^0(\mathbb{R}_+ \times \overline{\Omega})$ if and only if $\forall y \in \partial \Omega^-,$

\[ f_b^-(0,y) = f^{in}(y), \quad \frac{\partial f_b^-}{\partial t}(t,y)+v \cdot \nabla f^{in}(y)=0. \]

\paragraph{}

The expression of the solution is then given by: 
\[
\boxed{
	f(t,x)= 
	\begin{cases}
		f^{in}(x-tv) \quad si~ t \leq \tau_x ,\\
		f_b^-(t-\tau_x,x-\tau_xv) \quad si~ t > \tau_x ,
	\end{cases}
}
\]

With $\tau_x= \inf \{t \geq 0~ | ~ x-tv \notin \overline{\Omega}\}$

\paragraph{With Absorption and Source Term}

\paragraph{}
We still consider $\Omega$ to be convex, with $f_b^- \in \mathcal{C}^1(\mathbb{R}_+ \times \partial \Omega^-)$, $f^{in} \in \mathcal{C}^1(\overline{\Omega})$, $a$, and $S$ $\in \mathcal{C}^1(\mathbb{R}_+ \times \overline{\Omega})$. If we assume that $\forall y \in \partial \Omega^-$,

\[
\begin{aligned}
	f_b^-(0,y) &= f^{in}(y), \\
	\frac{\partial f_b^-}{\partial t}(0,y) + v \cdot \nabla_x f^{in}(y) + a(0,y)f^{in}(y)&=S(0,y)
\end{aligned}
\]



In this case, the problem becomes:

\[
\begin{cases}
	\frac{\partial f}{\partial t}(t,x)+v \cdot \nabla_x f(t,x)+ a(t,x)f(t,x)=S(t,x), \quad x \in \mathbb{R}^N, \quad t>0.\\
	\restriction{f}{\partial \Omega}= f_b^-\\
	\restriction{f}{t=0}= f^{in}(x)
\end{cases}
\]

has a unique solution $f \in \mathcal{C}^1(\mathbb{R}_+ \times \Omega)$ and $f \in \mathcal{C}^0(\mathbb{R}_+ \times \overline{\Omega})$. The expression for $f$ is then:


\[
\boxed{
	\begin{aligned}
		f(t,x) &= \mathbf{1}_{t \leq \tau_x} f^{in} (x-tv) \exp\left(-\int_0^t a(s,x+(s-t)v)ds\right) \\
		&\quad + \mathbf{1}_{t > \tau_x} f_b^- (t-\tau_x,x-\tau_x v) \exp\left(-\int_{t-\tau_x}^t a(s,x+(s-t)v)ds\right) \\
		&\quad + \int_{t-\tau_x}^t \exp\left(-\int_{s}^{t}a(\tau,x+(\tau-t)v)d\tau\right) S(s,x+(s-t)v)ds
\end{aligned}}
\]


\subsubsection{Generalized Solutions}

In this section, $\Omega$ is not necessarily convex. This implies that the continuity of potential solutions is not guaranteed. However, in practice, these cases must be addressed. Below, we define the concept of generalized solutions, which allows us to handle this issue without resorting to the theory of distributions.


\paragraph{Definition}

Let $\Omega$ be an open set with a $\mathcal{C}^1$ boundary in $\mathbb{R}^n$, and let $a$ and $S$ be continuous on $\mathbb{R}_+ \times \Omega$. We say that $f$ is a generalized solution of the problem

$$\frac{\partial f}{\partial t}(t,x)+v \cdot \nabla_x f(t,x)+ a(t,x)f(t,x)=S(t,x), \quad x \in \mathbb{R}^N, \quad t>0.\\$$

if and only if the function

$s \mapsto f(t+s,x+sv)$ is of class $\mathcal{C}^1$ almost everywhere in $(t,x) \in \mathbb{R}_+ \times \Omega$ and satisfies

$\frac{d}{ds}f(t+s,x+sv) +a(t+s,x+sv)f(t+s,x+sv) = S(t+s,x+sv) \quad (t+s,x+sv) \in \mathbb{R}_+^* \times \Omega$.


\paragraph{}



Let $\Omega$ be an open set with a $\mathcal{C}^1$ boundary in $\mathbb{R}^n$, and let $a$ and $S$ be continuous on $\mathbb{R}_+ \times \Omega$. Let $f_b^- \in \mathbf{L}_{loc}^{\infty}(\mathbb{R}_+ \times \partial \Omega^-)$ and $f^{in} \in \mathbf{L}_{loc}^{\infty}(\overline{\Omega})$.

In this case, the problem:

\[
\begin{cases}
	\frac{\partial f}{\partial t}(t,x)+v \cdot \nabla_x f(t,x)+ a(t,x)f(t,x)=S(t,x), \quad x \in \mathbb{R}^N, \quad t>0.\\
	\restriction{f}{\partial \Omega}= f_b^-\\
	\restriction{f}{t=0}= f^{in}(x)
\end{cases}
\]

has a unique generalized solution. The formula for this solution is the same as in the previous case, i.e., almost everywhere in $(t,x) \in \mathbb{R}_+ \times \Omega$, we have:

\[
\boxed{
	\begin{aligned}
		f(t,x) &= \mathbf{1}_{t \leq \tau_x} f^{in} (x-tv) \exp\left(-\int_0^t a(s,x+(s-t)v)ds\right) \\
		&\quad + \mathbf{1}_{t > \tau_x} f_b^- (t-\tau_x,x-\tau_x v) \exp\left(-\int_{t-\tau_x}^t a(s,x+(s-t)v)ds\right) \\
		&\quad + \int_{t-\tau_x}^t \exp\left(-\int_{s}^{t}a(\tau,x+(\tau-t)v)d\tau\right) S(s,x+(s-t)v)ds
\end{aligned}}
\]







\subsection{Linear Boltzmann Equation}

In this section, we will focus on the actual Boltzmann equation. This integro-differential equation is described in \ref{Boltzmann}.

\begin{equation} \label{Boltzmann}
	\frac{\partial f}{\partial t}(t,x,v)+v \cdot \nabla_x f(t,x,v) + a(t,x)f(t,x,v) = \int_{\mathbb{R}^n}k(t,x,v,w)f(t,x,w)d\mu(w)+Q(t,x,v).
\end{equation}

To simplify notations, this equation will be written as: 
\[ \frac{\partial f}{\partial t}+v \cdot \nabla_x f +af =   \mathcal{K} f +Q \]

Here, $v$ and $k$ are given continuous functions. The range of possibilities for the velocity $v$ is very diverse due to the variety of physical problems that can be considered. Therefore, the notation $d\mu(w)$ encompasses all these possibilities by incorporating a weight function. The major difference from the transport equation is that, this time, we consider exchange phenomena, and thus, $v$ cannot be seen as a simple parameter.

In this section, we cannot highlight the exact expression of the solutions because the method of characteristics yields another integral equation involving the solution.




\paragraph{Definition}

Let $Q$ be continuous on $]0,T[\times \Omega \times \mathbb{R}^n$. We say that $f$ is a generalized solution of the problem, continuous on $]0,T[\times \Omega \times \mathbb{R}^n$,


$$\frac{\partial f}{\partial t}+v \cdot \nabla_x f +af = \mathcal{K} f +Q\\$$

\paragraph{}
if and only if, for all $(t,x,v) \in ]0,T[\times \Omega \times \mathbb{R}^n$, the function 

\[ s \mapsto f(t+s,x+sv,v) \]

is of class $\mathcal{C}^1$ for all $x+sv \in \Omega$ and satisfies, for all $s \in \mathbb{R}$ such that $x +sv \in \Omega$,

$\frac{d}{ds}f(t+s,x+sv,v) +a(t+s,x+sv,v)f(t+s,x+sv,v) = (\mathcal{K}f+ Q)(t+s,x+sv,v).$

\subsubsection{Cauchy Problem}
\paragraph{}



Let $f^{in} \in \mathbf{C}_b(\mathbb{R}^n \times \mathbb{R}^n)$ and $Q \in \mathbf{C}_b(]0,T[ \times \mathbb{R}^n \times \mathbb{R}^n)$. Also, let $0 \leq a \in \mathbf{C}_b(]0,T[ \times \mathbb{R}_x^n \times \mathbb{R}_v^n)$ and $0 \leq k \in \mathbf{C}_b(]0,T[ \times \mathbb{R}_x^n \times \mathbb{R}_v^n \times \mathbb{R}_w^n)$. The notation $\mathbf{C}_b(X)$ denotes the set of bounded continuous functions on $X$ with values in $\mathbb{R}$.


We assume that:  \[ \underset{(t,x,v) \in ]0,T[ \times \mathbb{R}^n \times \mathbb{R}^n}{\sup} \int_{\mathbb{R}^n} k(t,x,v,w) \, d\mu(w) < +\infty \]

\paragraph{}

Then the problem is:

\[
\begin{cases}
	\frac{\partial f}{\partial t}+v \cdot \nabla_x f +af =   \mathcal{K} f +Q \quad (t,x,v) \in ]0,T[ \times \mathbb{R}^n \times \mathbb{R}^n\\
	\restriction{f}{t=0}= f^{in}
\end{cases}
\]
has a unique generalized solution $f \in \mathbf{C}_b(]0,T[ \times \mathbb{R}^n \times \mathbb{R}^n)$.

\subsubsection{Boundary Value Problem}

We define the set $\Gamma_-$ as $\Gamma_-= \{(x,v) \in \partial \Omega \times \mathbb{R}^n ~ | ~ v\cdot n_x<0\}$

Let $f^{in} \in \mathbf{C}_b(\overline{\Omega} \times \mathbb{R}^n)$, $f_b^- \in \mathbf{C}_b([0,T] \times \Gamma_-)$ such that $f_b^-(0,y,v)=f^{in}(y,v) \quad \forall (y,v) \in \Gamma_-$

Let $Q \in \mathbf{C}_b([0,T] \times \overline{\Omega} \times \mathbb{R}^n)$.

Let $0 \leq a \in \mathbf{C}_b(\mathbb{R}_+ \times \Omega_x \times \mathbb{R}_v^n)$ and $0 \leq k \in \mathbf{C}_b(\mathbb{R}_+ \times \Omega_x^n \times \mathbb{R}_v^n \times \mathbb{R}_w^n)$.

Then the problem

\[
\begin{cases}
	\frac{\partial f}{\partial t}+v \cdot \nabla_x f +af =   \mathcal{K} f +Q \quad (t,x,v) \in ]0,T[ \times \Omega \times \mathbb{R}^n\\
	\restriction{f}{t=0}= f^{in}\\
	\restriction{f}{\Gamma^-}=f_b^-
\end{cases}
\]

has a unique generalized solution $f \in \mathbf{C}_b([0,T] \times \Omega \times \mathbb{R}^n)$.

\section{The use of Monte Carlo methods to solve the linear Boltzmann equation}

To facilitate the introduction of a numerical Scheme let introduce new notation.
The transport equation in an infinite medium with its corresponding deterministic collisional component can be expressed as:
\begin{equation}
	\partial _t u(x,t,\textbf{v}) + \textbf{v} \cdot \nabla u(x,t,\textbf{v}) + v\sigma_t (x,t,\textbf{v})u(x,t,\textbf{v})= v\sigma_s(x,t,\textbf{v})\int P (x,t,\textbf{v},\textbf{v}')u(x,t,\textbf{v}')d\textbf{v}' \label{ref11}
\end{equation}

Where 
\begin{equation*}
	\sigma_s (x,t,\textbf{v})= \int \sigma_s (x,t,\textbf{v},\textbf{v}')d\textbf{v}', \quad  P (x,t,\textbf{v},\textbf{v}')=
	\frac{\sigma_s (x,t,\textbf{v},\textbf{v}')}{\sigma_s (x,t,\textbf{v})}
\end{equation*}

The approach involves a series of variable changes. The initial step involves re-expressing the transport equation \ref{ref11} with respect to a characteristic $x + vt$. As a result, it transforms into:

\begin{equation}
	\partial _s u(x+\textbf{v}s,s,\textbf{v}) = -v\sigma_t (x+\textbf{v}s,s,\textbf{v})u(x+\textbf{v}s,s,\textbf{v}) + v\sigma_s(x+\textbf{v}s,s,\textbf{v})\int P (x+\textbf{v}s,s,\textbf{v},\textbf{v}')u(x+\textbf{v}s,s,\textbf{v}')d\textbf{v}'
\end{equation}

Let's multiply both sides of the equation by:
\begin{equation*}
	e^{\int _0^s v\sigma_t (x + \textbf{v}\alpha,\alpha, v) d\alpha}
\end{equation*}
Following that, we obtain
\begin{equation*}
	\partial _s [u(x+\textbf{v}s,s,\textbf{v})e^{\int _0^s v\sigma_t (x + \textbf{v}\alpha,\alpha, v) d\alpha}] = e^{\int _0^s v\sigma_t (x + \textbf{v}\alpha,\alpha, v) d\alpha} v\sigma_s(x+\textbf{v}s,s,\textbf{v})\int P (x+\textbf{v}s,s,\textbf{v},\textbf{v}')u(x+\textbf{v}s,s,\textbf{v}')d\textbf{v}' \label{ref12}
\end{equation*}
We get after integrating the equation \ref{ref12} between [0, t]:

\begin{multline}
	u(x+\textbf{v}t,t,\textbf{v}) = u_0(x, \textbf{v}) \exp\left(- \int_{0}^{t} v\sigma_t\left(x + \textbf{v} \alpha, \alpha, \textbf{v}\right) d\alpha\right) \\
	+ \int_{0}^{t} \int v\sigma_s\left(x + \textbf{v}s, s, \textbf{v}\right) u\left(x + \textbf{v}s, s, \textbf{v}'\right) e^{- \int_s^t v\sigma_t\left(x + \textbf{v} \alpha, \textbf{v}\right) d\alpha} P\left(x + \textbf{v} s, s, \textbf{v}, \textbf{v}'\right) d\textbf{v}'ds 
\end{multline}

We obtain:

\begin{multline}
	u(x,t,\textbf{v}) = u_0(x - \textbf{v}t, \textbf{v}) \exp\left(- \int_{0}^{t} v\sigma_t\left(x - \textbf{v}(t - \alpha), \alpha, \textbf{v}\right) d\alpha\right) \\
	+ \int_{0}^{t} \int v\sigma_s\left(x - \textbf{v}(t - s), s, \textbf{v}\right) u\left(x - \textbf{v}(t - s), s, \textbf{v}'\right) e^{- \int_s^t v\sigma_t\left(x - \textbf{v}(t - \alpha), \textbf{v}\right) d\alpha} P\left(x - \textbf{v}(t - s), s, \textbf{v}, \textbf{v}'\right) d\textbf{v}'ds \label{ref1}
\end{multline}
We also have:
\begin{multline}
	\exp\left(- \int_{0}^{t} v\sigma_t\left(x - \textbf{v}(t - \alpha), \alpha, \textbf{v}\right) d\alpha\right) = \exp\left(- \int_{0}^{t} v\sigma_t\left(x - \textbf{v} \alpha,t- \alpha, \textbf{v}\right) d\alpha\right) \\ = \int _t^\infty  v\sigma_t\left(x - \textbf{v} s,t- s, \textbf{v}\right)
	\exp\left(- \int_{0}^{s} v\sigma_t\left(x - \textbf{v} \alpha,t- \alpha, \textbf{v}\right) d\alpha\right) ds
\end{multline}
Then the integral representation of \ref{ref11} is provided by:

\begin{multline}
	u(x,t,\textbf{v}) =  \int _t^\infty  u_0(x - \textbf{v}t, \textbf{v}) v\sigma_t\left(x - \textbf{v} s,t- s, \textbf{v}\right)
	\exp\left(- \int_{0}^{s} v\sigma_t\left(x - \textbf{v} \alpha,t- \alpha, \textbf{v}\right) d\alpha\right) ds\\
	+ \int_{0}^{t} \int v\sigma_s\left(x - \textbf{v}(t - s), s, \textbf{v}\right) u\left(x - \textbf{v}(t - s), s, \textbf{v}'\right) e^{- \int_s^t v\sigma_t\left(x - \textbf{v}(t - \alpha), \textbf{v}\right) d\alpha} P\left(x - \textbf{v}(t - s), s, \textbf{v}, \textbf{v}'\right) d\textbf{v}'ds \label{ref13}
\end{multline}


\section{Semi analog MC scheme}

Developing a Monte Carlo scheme involves introducing a group of random variables along with their associated probability measure to express \ref{ref13} as an expectation. The selection of this set of random variables is not unique, resulting in various Monte Carlo schemes with distinct properties. In this project we used one of them, the so-called Semi-analog scheme. The algorithm 1 describe this method.


\paragraph{}
This method is described as backward. In fact, we first initialize $S_p$, the remaining lifetime of the particle, to $t$, and it will decrease to zero. Here, we adopt a general form for the boundary conditions. In the program we developed, we have tried various conditions: Periodic boundary conditions (which are not truly significant in a physical problem but can be interpreted as an infinite environment), and Elastic boundary conditions.

At line 11, we sample $\tau$ using the function: $\tau = \frac{-\ln(\mathcal{U}(0,1))}{\sigma_t \sqrt{v^2}}$. If $\tau$ is greater than $s_p$, then the particle is moved, its lifetime is set to zero, and its contribution is updated. Otherwise, the particle is moved, its speed is sampled, its lifetime is updated, and then its weight is changed.






\begin{algorithm}
	\caption{Semi analog scheme algorithm}
	\begin{algorithmic}[1]
		\State $u(x, t, v) \gets 0$
		\For{$p = 1$ to $N_{MC}$}
		\State $s_p \gets t$ 
		\State $x_p \gets x$
		\State $v_p \gets v$
		\State $w_p \gets \frac{1}{N_{MC}}$
		\While{$s_p > 0$ AND $w_p > 0$}
		\If{$x_p \notin D$}
		\State apply\_boundary\_conditions$(x_p, s_p, v_p)$
		\EndIf
		\State Sample $\tau$.
		\If{$\tau > s_p$}



		\State $x_p \gets - v_p s_p$

		\State $s_p \gets 0$

		\State $u(x, t, v) \gets w_p \times u_0(x_p, v_p)$
		\Else

		\State $x_p \gets - v_p \tau$
		\State Sample the velocity of particle p from $P(x_p, s_p, v_p, dv')$
		\State $v_p \gets v'$

		\State $s_p \gets - \tau$

		\State $w_p \gets \frac{\sigma(x_p, s_p, v')}{\sigma(x_p, s_p, v_p)} \times w_p$
		\EndIf
		\EndWhile
		\EndFor
	\end{algorithmic}
\end{algorithm}




\section{Numerical tests/results}

à voir

strucutre code
mise en place bnr
differents test dispo cf les prties récédentes

resultats

\section{Conclusion}


	
	
\end{document}