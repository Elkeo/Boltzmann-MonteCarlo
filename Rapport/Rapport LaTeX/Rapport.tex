\documentclass[a4paper, 11pt]{article}
\usepackage{inputenc}
\usepackage{amsmath}
\usepackage{graphicx}
\usepackage[T1]{fontenc}
\usepackage{babel}
\usepackage{hyperref}    %pour indexer la table des matières 
\hypersetup{pdfborder=1 1 1}
\usepackage{float} 
\usepackage{url}
\usepackage{caption}
\usepackage{multirow}
\usepackage{multicol}
\usepackage{listings}
\usepackage{subcaption}
\usepackage{empheq}
\usepackage{fancyhdr}
\usepackage{booktabs}
\usepackage{amssymb}
\usepackage{xcolor}
\usepackage{setspace}
\usepackage{enumitem}
\setlength{\hoffset}{-18pt}         
\setlength{\oddsidemargin}{0pt} % Marge gauche sur pages impaires
\setlength{\evensidemargin}{9pt} % Marge gauche sur pages paires
\setlength{\marginparwidth}{54pt} % Largeur de note dans la marge
\setlength{\textwidth}{481pt} % Largeur de la zone de texte (17cm)
\setlength{\voffset}{-18pt} % Bon pour DOS
\setlength{\marginparsep}{7pt} % Séparation de la marge
\setlength{\topmargin}{0pt} % Pas de marge en haut
\setlength{\headheight}{13pt} % Haut de page
\setlength{\headsep}{10pt} % Entre le haut de page et le texte
\setlength{\footskip}{27pt} % Bas de page + séparation
\setlength{\textheight}{708pt} % Hauteur de la zone de texte (25cm)
%Traits en haut et en bas des pages
\usepackage{fancyhdr}
\pagestyle{fancy}
\fancyhead[L]{}
\fancyhead[R]{}
\renewcommand\footrulewidth{1pt}
\renewcommand\footrulewidth{1pt}
\fancyfoot[L]{\tiny Project Report}
\usepackage{lastpage}
\fancyfoot[R]{\tiny January 2024}
\fancyfoot[C]{\textbf{Page \thepage/\pageref{LastPage}}}
\lstnewenvironment{mathematicacode}[1][]
{
	\lstset{
		language=Mathematica,
		basicstyle=\small\ttfamily,
		numbers=left,
		numberstyle=\tiny,
		numbersep=5pt,
		frame=single,
		frameround=tttt,
		framexleftmargin=5pt,
		#1
	}
}
{}

\begin{document} 
	
	%----------------------------------------------------------------------------------------
	%	TITLE PAGE
	%----------------------------------------------------------------------------------------
	
	\begin{titlepage} % Suppresses displaying the page number on the title page and the subsequent page counts as page 1
		\newcommand{\HRule}{\rule{\linewidth}{0.5mm}} % Defines a new command for horizontal lines, change thickness here
		
		\centering % Centre everything on the page
		
		%------------------------------------------------
		%	Headings
		%------------------------------------------------
		
		\textsc{\LARGE ENSEIRB-MATMECA}\\[1cm] % Main heading such as the name of your university/college
		
		
		\begin{figure}[H]
			\begin{center}
				\includegraphics[scale=4]{logo ecole.jpg}
			\end{center}
		\end{figure}
		
		%\textsc{\large Minor Heading}\\[0.5cm] % Minor heading such as course title
		
		%------------------------------------------------
		%	Title
		%------------------------------------------------
		
		\vspace{1cm}
		
		\HRule\\[0.4cm]
		
		{\huge\bfseries Solving the linear Boltzmann equation using Monte Carlo methods }\\[0.4cm] % Title of your document
		
		\HRule\\[1.5cm]
		
		{\huge\bfseries }
		
		\bigskip
		\bigskip
		
		\centering
		\Large{Project lead by :} \\
		\Large{C. Aumonier, A. Boucher, G. Doyen, K. El Maddah, G. Rodiere}
		
		%------------------------------------------------
		%	Date
		%------------------------------------------------
		
		\vfill\vfill\vfill % Position the date 3/4 down the remaining page
		{\Large  Project supervised by :}\\
		\Large{G. Poëtte}
		
		\vspace{0,5cm}
		
		{\large January 2024} % Date, change the \today to a set date if you want to be precise
		
		%------------------------------------------------
		%	Logo
		%------------------------------------------------
		
		%\vfill\vfill
		%\includegraphics[width=0.2\textwidth]{placeholder.jpg}\\[1cm] % Include a department/university logo - this will require the graphicx package
		
		%----------------------------------------------------------------------------------------
		
		\vfill % Push the date up 1/4 of the remaining page
		
	\end{titlepage}
	
\tableofcontents

\newpage
	
\section{Introduction}

**Gabriel

\section{Transport equations}

insérer la partie écrite par Clément
**Gabriel traduction

\section{The use of Monte Carlo methods to solve the linear Boltzmann equation}

insérer la partie écrite par Khaoula
**Khaoula

\section{Semi analog MC scheme}

**Khaoula

\section{Numerical tests/results}

à voir

\section{Conclusion}


	
	
\end{document}